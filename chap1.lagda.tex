\documentclass{article}

\usepackage[paperwidth=5.5in,paperheight=8.5in,margin=0.5in,footskip=.25in]{geometry}
\usepackage{fontspec}
\usepackage{mathtools}
\usepackage{enumitem}
\usepackage{unicode-math}
\usepackage{fancyvrb}
\usepackage{syntax}
\usepackage{tikz}

\DefineVerbatimEnvironment{code}{Verbatim}{baselinestretch=.8, samepage=true}

\setmainfont{Garamond Premier Pro}[Contextuals=AlternateOff]
\setmathfont{Libertinus Math}[Scale=MatchUppercase]
\setmonofont{JuliaMono}[Scale=0.7]

\setlength{\parindent}{1em}
\setlist{noitemsep}

\newcommand{\ttx}{\texttt}

\begin{document}
\noindent
I'm going to learn some agda!

\begin{code}
data Greeting : Set where
    hello : Greeting

greet : Greeting
greet = hello
\end{code}

\noindent
Defining the natural numbers:

\begin{code}
data Nat : Set where
    zero : Nat
    suc : Nat → Nat

{-# BUILTIN NATURAL Nat #-}

_+_ : Nat → Nat → Nat
zero + y = y
suc x + y = suc (x + y)

infixl 6  _+_
\end{code}

\noindent
\textsc{Exercise 1.1}: Define the function \ttx{halve : Nat → Nat} that computes the result of dividing the given number by 2 (rounded down). Test your definition by evaluating it for several concrete inputs.

\begin{code}
halve : Nat → Nat
halve 0 = 0
halve 1 = 0
halve (suc (suc n)) = halve n + 1
\end{code}

\noindent
\textsc{Exercise 1.2}: Define the function \ttx{_*_ : Nat → Nat → Nat} for multiplication of two natural numbers.

\begin{code}
_*_ : Nat → Nat → Nat
0 * y = 0
suc x * y = y + (x * y)

infixl 7  _*_
\end{code}

\noindent
\textsc{Exercise 1.3}: Define the type Bool with constructors true and false, and define the functions for negation \verb$not : Bool → Bool$, conjunction \verb$_&&_ : Bool → Bool → Bool$, and disjunction \verb$_||_ : Bool → Bool → Bool$ by pattern matching.

\begin{code}
data Bool : Set where
    true : Bool
    false : Bool

not : Bool → Bool
not true = false
not false = true
\end{code}

\begin{code}
id : {A : Set} → A → A
id x = x

data List (A : Set) : Set where
    [] : List A
    _::_ : A → List A → List A

infixr 5 _::_

data _×_ (A B : Set) : Set where
    _,_ : A → B → A × B

fst : {A B : Set} → A × B → A
fst (x , _) = x

snd : {A B : Set} → A × B → B
snd (_ , y) = y
\end{code}

\noindent
\textsc{Exercise 1.4}:

\begin{code}
length : {A : Set} → List A → Nat
length [] = 0
length (x :: xs) = suc (length xs)

_++_ : {A : Set} → List A → List A → List A
[] ++ ys = ys
(x :: xs) ++ ys = x :: (xs ++ ys)

map : {A B : Set} → (A → B) → List A → List B
map f [] = []
map f (x :: xs) = f x :: map f xs
\end{code}

\noindent
\textsc{Exercise 1.5}:

\begin{code}
data Maybe (A : Set) : Set where
    nothing : Maybe A
    just : A → Maybe A

lookup : {A : Set} → List A → Nat → Maybe A
lookup [] _ = nothing
lookup (x :: xs) zero = just x
lookup (x :: xs) (suc i) = lookup xs i
\end{code}

\begin{code}
data Vec (A : Set) : Nat → Set where
    [] : Vec A 0
    _::_ : {n : Nat} → A → Vec A n → Vec A (suc n)

replicate : {A : Set} → (n : Nat) → A → Vec A n
replicate zero x = []
replicate (suc n) x = x :: replicate n x
\end{code}

\noindent
\textsc{Exercise 2.1}:

\begin{code}
downFrom : (n : Nat) → Vec Nat n
downFrom zero = []
downFrom (suc x) = x :: downFrom x
\end{code}

\begin{code}
_++Vec_ : {A : Set} {m n : Nat}
    → Vec A m → Vec A n → Vec A (m + n)
[] ++Vec ys = ys
(x :: xs) ++Vec ys = x :: (xs ++Vec ys)

head : {A : Set} {n : Nat} → Vec A (suc n) → A
head (x :: _) = x
\end{code}

\noindent
\textsc{Exercise 2.2}:

\begin{code}
tail : {A : Set} {n : Nat} → Vec A (suc n) → Vec A n
tail (_ :: xs) = xs
\end{code}

\noindent
\textsc{Exercise 2.3}:

\begin{code}
dotProduct : {n : Nat} → Vec Nat n → Vec Nat n → Nat
dotProduct [] [] = 0
dotProduct (x :: xs) (y :: ys) = x * y + dotProduct xs ys
\end{code}

\begin{code}
data Fin : Nat → Set where
    zero : {n : Nat} → Fin (suc n)
    suc : {n : Nat} → Fin n → Fin (suc n)

zero3 : Fin 3
zero3 = zero

lookupVec : {A : Set} {n : Nat} → Vec A n → Fin n → A
lookupVec (x :: xs) zero = x
lookupVec (x :: xs) (suc i) = lookupVec xs i
\end{code}

\noindent
\textsc{Exercise 2.4}:

\begin{code}
putVec : {A : Set} {n : Nat} → Fin n → A → Vec A n → Vec A n
putVec zero x (_ :: xs) = x :: xs
putVec (suc i) x (x₁ :: xs) = x₁ :: putVec i x xs
\end{code}

\begin{code}
data Σ (A : Set) (B : A → Set) : Set where
    _,_ : (x : A) → B x → Σ A B

fstΣ : {A : Set} {B : A → Set} → Σ A B → A
fstΣ (x , _) = x

sndΣ : {A : Set} {B : A → Set} → (z : Σ A B) → B (fstΣ z)
sndΣ (x , y) = y

_×'_ : (A B : Set) → Set
A ×' B = Σ A (λ _ → B)
\end{code}

\noindent
\textsc{Exercise 2.5}:

\begin{code}
fromProd : {A B : Set} → A × B → A ×' B
fromProd (x , y) = x , y

toProd : {A B : Set} → A ×' B → A × B
toProd (x , y) = x , y
\end{code}

\begin{code}
List' : (A : Set) → Set
List' A = Σ Nat (Vec A)
\end{code}

\noindent
\textsc{Exercise 2.6}:

\begin{code}
[]' : {A : Set} → List' A
[]' = 0 , []

_::'_ : {A : Set} → A → List' A → List' A
x ::' (n , xs) = suc n , (x :: xs)

fromList : {A : Set} → List A → List' A
fromList [] = []'
fromList (x :: xs) = x ::' fromList xs

fromList' : {A : Set} → List' A → List A
fromList' (0 , []) = []
fromList' (suc n , (x :: xs)) = x :: (fromList' (n , xs))
\end{code}

\noindent
\textsc{Exercise 3.1}:

\begin{code}
data Either (A : Set) (B : Set) : Set where
    left : A → Either A B
    right : B → Either A B

cases : {A B C : Set} → Either A B → (A → C) → (B → C) → C
cases (left x) f _ = f x
cases (right x) _ f = f x
\end{code}

\begin{code}
data ⊤ : Set where
    tt : ⊤

data ⊥ : Set where

absurd : {A : Set} → ⊥ → A
absurd ()
\end{code}

\noindent
\textsc{Exercise 3.2}
\begin{itemize}
\item If \textit{A} then (\textit{B} implies \textit{A})
\begin{code}
p1 : {A B : Set} → A → (B → A)
p1 x _ = x
\end{code}
\item If (\textit{A} and \textit{true}) then (\textit{A} or \textit{false})
\begin{code}
p2 : {A : Set} → (A × ⊤) → (Either A ⊥)
p2 (x , tt) = left x
\end{code}
\item If \textit{A} implies (\textit{B} implies \textit{C}), then (\textit{A} and \textit{B}) implies \textit{C}.
\begin{code}
p3 : {A B C : Set} → (A → (B → C)) → ((A × B) → C)
p3 f (x , y) = f x y
\end{code}
\item If \textit{A} and (\textit{B} or \textit{C}), then either (\textit{A} and \textit{B}) or (\textit{A} and \textit{C}).
\begin{code}
p4 : {A B C : Set} → (A × (Either B C)) → (Either (A × B) (A × C))
p4 (x , left y) = left (x , y)
p4 (x , right z) = right (x , z)
\end{code}
\item If A implies C and B implies D, then (A and B) implies (C and D).
\begin{code}
p5 : {A B C D : Set} → ((A → C) × (B → D)) → ((A × B) → (C × D))
p5 (f , g) (x , y) = f x , g y
\end{code}
\end{itemize}

\begin{code}
proof3 : {P Q R : Set} → (Either P Q → R) → (P → R) × (Q → R)
proof3 f = (λ x → f (left x)) , (λ x → f (right x))
\end{code}

\noindent
\textsc{Exercise 3.3}: Write a function of type \verb!{P : Set} → (Either P (P → ⊥) → ⊥) → ⊥!.

Assuming \verb!(Either P (P → ⊥) → ⊥)! then \texttt{proof3} above says that \verb!P → ⊥! and \verb!(P → ⊥) → ⊥!. Applying \verb!(P → ⊥) → ⊥! to \verb!P → ⊥! results in \verb!⊥! which proves the theorem.

\begin{code}
constructive-P-or-not-P : {P : Set} → (Either P (P → ⊥) → ⊥) → ⊥
constructive-P-or-not-P f = (λ notx → f (right notx)) (λ x → f (left x))
\end{code}

\noindent Some even stuff:

\begin{code}
data IsEven : Nat → Set where
    zeroIsEven : IsEven zero
    sucsucIsEven : {n : Nat} → IsEven n → IsEven (suc (suc n))

6-is-even : IsEven 6
6-is-even = sucsucIsEven (sucsucIsEven (sucsucIsEven zeroIsEven))

7-is-even : IsEven 7 → ⊥
7-is-even (sucsucIsEven (sucsucIsEven (sucsucIsEven ())))

data IsTrue : Bool → Set where
    TrueIsTrue : IsTrue true

_=Nat_ : Nat → Nat → Bool
zero =Nat zero = true
suc x =Nat suc y = x =Nat y
_ =Nat _ = false

length-is-3 : IsTrue (length (1 :: 2 :: 3 :: []) =Nat 3)
length-is-3 = TrueIsTrue
\end{code}

\begin{code}
double : Nat → Nat
double zero = zero
double (suc n) = suc (suc (double n))

double-is-even : (n : Nat) → IsEven (double n)
double-is-even zero = zeroIsEven
double-is-even (suc n) = sucsucIsEven (double-is-even n)
\end{code}

\begin{code}
n-equals-n : (n : Nat) → IsTrue (n =Nat n)
n-equals-n zero = TrueIsTrue
n-equals-n (suc n) = n-equals-n n
\end{code}

\end{document}
 