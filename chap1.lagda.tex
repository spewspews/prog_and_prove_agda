\documentclass{article}

\usepackage[paperwidth=5.5in,paperheight=8.5in,margin=0.5in,footskip=.25in]{geometry}
\usepackage{fontspec}
\usepackage{mathtools}
\usepackage{enumitem}
\usepackage{unicode-math}
\usepackage{fancyvrb}
\usepackage{syntax}
\usepackage{tikz}

\DefineVerbatimEnvironment{code}{Verbatim}{baselinestretch=.8, samepage=true}

\setmainfont{Garamond Premier Pro}[Contextuals=AlternateOff]
\setmathfont{Libertinus Math}[Scale=MatchUppercase]
\setmonofont{Monaspace Argon}[Scale=0.7]

\setlength{\parindent}{1em}
\setlist{noitemsep}

\newcommand{\txt}{\texttt}

\begin{document}
\noindent
I'm going to learn some agda!

\begin{code}
data Greeting : Set where
    hello : Greeting

greet : Greeting
greet = hello
\end{code}

\noindent
Defining the natural numbers:

\begin{code}
data Nat : Set where
    zero : Nat
    suc : Nat → Nat

_+_ : Nat → Nat → Nat
zero + y = y
suc x + y = suc (x + y)

{-# BUILTIN NATURAL Nat #-}
\end{code}

\noindent
\textsc{Exercise 1.1} Define the function \txt{halve : Nat → Nat} that computes the result of dividing the given number by 2 (rounded down). Test your definition by evaluating it for several concrete inputs.

\begin{code}
halve : Nat → Nat
halve 0 = 0
halve 1 = 0
halve (suc (suc n)) = halve n + 1
\end{code}

\noindent
\textsc{Exercise 1.2} Define the function \txt{_*_ : Nat → Nat → Nat} for multiplication of two natural numbers.

\begin{code}
_*_ : Nat → Nat → Nat
0 * y = 0
suc x * y = y + (x * y)
\end{code}

\end{document}