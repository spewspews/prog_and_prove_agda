\documentclass{article}

\usepackage[paperwidth=5.5in,paperheight=8.5in,margin=0.5in,footskip=.25in]{geometry}
\usepackage{fontspec}
\usepackage{mathtools}
\usepackage{enumitem}
\usepackage{unicode-math}
\usepackage{fancyvrb}
\usepackage{syntax}
\usepackage{tikz}

\DefineVerbatimEnvironment{code}{Verbatim}{baselinestretch=.8, samepage=true}

\setmainfont{Garamond Premier Pro}[Contextuals=AlternateOff, Numbers=OldStyle]
\setmathfont{Libertinus Math}[Scale=MatchUppercase]
\setmonofont{JuliaMono}[Scale=0.7]

\setlength{\parindent}{1em}
\setlist{noitemsep}

\newcommand{\ttx}{\texttt}

\begin{document}
\noindent
I'm going to learn some agda!

\begin{code}
data Greeting : Set where
    hello : Greeting

greet : Greeting
greet = hello
\end{code}

\noindent
Defining the natural numbers:

\begin{code}
data Nat : Set where
    zero : Nat
    suc : Nat → Nat

{-# BUILTIN NATURAL Nat #-}

_+_ : Nat → Nat → Nat
zero + y = y
suc x + y = suc (x + y)

infixl 6  _+_
\end{code}

\noindent
\textsc{Exercise 1.1}: Define the function \ttx{halve : Nat → Nat} that computes the result of dividing the given number by 2 (rounded down). Test your definition by evaluating it for several concrete inputs.

\begin{code}
halve : Nat → Nat
halve 0 = 0
halve 1 = 0
halve (suc (suc n)) = halve n + 1
\end{code}

\noindent
\textsc{Exercise 1.2}: Define the function \ttx{_*_ : Nat → Nat → Nat} for multiplication of two natural numbers.

\begin{code}
_*_ : Nat → Nat → Nat
0 * y = 0
suc x * y = y + (x * y)

infixl 7  _*_
\end{code}

\noindent
\textsc{Exercise 1.3}: Define the type Bool with constructors true and false, and define the functions for negation \verb$not : Bool → Bool$, conjunction \verb$_&&_ : Bool → Bool → Bool$, and disjunction \verb$_||_ : Bool → Bool → Bool$ by pattern matching.

\begin{code}
data Bool : Set where
    true : Bool
    false : Bool

not : Bool → Bool
not true = false
not false = true
\end{code}

\begin{code}
id : {A : Set} → A → A
id x = x

data List (A : Set) : Set where
    [] : List A
    _::_ : A → List A → List A

infixr 5 _::_

data _×_ (A B : Set) : Set where
    _,_ : A → B → A × B

fst : {A B : Set} → A × B → A
fst (x , _) = x

snd : {A B : Set} → A × B → B
snd (_ , y) = y
\end{code}

\noindent
\textsc{Exercise 1.4}:

\begin{code}
length : {A : Set} → List A → Nat
length [] = zero
length (x :: xs) = suc (length xs)

_++_ : {A : Set} → List A → List A → List A
[] ++ ys = ys
(x :: xs) ++ ys = x :: (xs ++ ys)

map : {A B : Set} → (A → B) → List A → List B
map f [] = []
map f (x :: xs) = f x :: map f xs
\end{code}

\noindent
\textsc{Exercise 1.5}:

\begin{code}
data Maybe (A : Set) : Set where
    nothing : Maybe A
    just : A → Maybe A

lookup : {A : Set} → List A → Nat → Maybe A
lookup [] _ = nothing
lookup (x :: xs) zero = just x
lookup (x :: xs) (suc i) = lookup xs i
\end{code}

\begin{code}
data Vec (A : Set) : Nat → Set where
    [] : Vec A zero
    _::_ : {n : Nat} → A → Vec A n → Vec A (suc n)

replicateVec : {A : Set} → (n : Nat) → A → Vec A n
replicateVec zero x = []
replicateVec (suc n) x = x :: replicateVec n x
\end{code}

\noindent
\textsc{Exercise 2.1}:

\begin{code}
downFrom : (n : Nat) → Vec Nat n
downFrom zero = []
downFrom (suc x) = x :: downFrom x
\end{code}

\begin{code}
_++Vec_ : {A : Set} {m n : Nat}
    → Vec A m → Vec A n → Vec A (m + n)
[] ++Vec ys = ys
(x :: xs) ++Vec ys = x :: (xs ++Vec ys)

head : {A : Set} {n : Nat} → Vec A (suc n) → A
head (x :: _) = x
\end{code}

\noindent
\textsc{Exercise 2.2}:

\begin{code}
tail : {A : Set} {n : Nat} → Vec A (suc n) → Vec A n
tail (_ :: xs) = xs
\end{code}

\noindent
\textsc{Exercise 2.3}:

\begin{code}
dotProduct : {n : Nat} → Vec Nat n → Vec Nat n → Nat
dotProduct [] [] = zero
dotProduct (x :: xs) (y :: ys) = x * y + dotProduct xs ys
\end{code}

\begin{code}
data Fin : Nat → Set where
    zero : {n : Nat} → Fin (suc n)
    suc : {n : Nat} → Fin n → Fin (suc n)

zero3 : Fin 3
zero3 = zero

lookupVec : {A : Set} {n : Nat} → Vec A n → Fin n → A
lookupVec (x :: xs) zero = x
lookupVec (x :: xs) (suc i) = lookupVec xs i
\end{code}

\noindent
\textsc{Exercise 2.4}:

\begin{code}
putVec : {A : Set} {n : Nat} → Fin n → A → Vec A n → Vec A n
putVec zero x (_ :: xs) = x :: xs
putVec (suc i) x (x₁ :: xs) = x₁ :: putVec i x xs
\end{code}

\begin{code}
data Σ (A : Set) (B : A → Set) : Set where
    _,_ : (x : A) → B x → Σ A B

fstΣ : {A : Set} {B : A → Set} → Σ A B → A
fstΣ (x , _) = x

sndΣ : {A : Set} {B : A → Set} → (z : Σ A B) → B (fstΣ z)
sndΣ (x , y) = y

_×'_ : (A B : Set) → Set
A ×' B = Σ A (λ _ → B)
\end{code}

\noindent
\textsc{Exercise 2.5}:

\begin{code}
fromProd : {A B : Set} → A × B → A ×' B
fromProd (x , y) = x , y

toProd : {A B : Set} → A ×' B → A × B
toProd (x , y) = x , y
\end{code}

\begin{code}
List' : (A : Set) → Set
List' A = Σ Nat (Vec A)
\end{code}

\noindent
\textsc{Exercise 2.6}:

\begin{code}
[]' : {A : Set} → List' A
[]' = zero , []

_::'_ : {A : Set} → A → List' A → List' A
x ::' (n , xs) = suc n , (x :: xs)

fromList : {A : Set} → List A → List' A
fromList [] = []'
fromList (x :: xs) = x ::' fromList xs

fromList' : {A : Set} → List' A → List A
fromList' (zero , []) = []
fromList' (suc n , (x :: xs)) = x :: (fromList' (n , xs))
\end{code}

\noindent
\textsc{Exercise 3.1}:

\begin{code}
data Either (A : Set) (B : Set) : Set where
    left : A → Either A B
    right : B → Either A B

cases : {A B C : Set} → Either A B → (A → C) → (B → C) → C
cases (left x) f _ = f x
cases (right x) _ f = f x
\end{code}

\begin{code}
data ⊤ : Set where
    tt : ⊤

data ⊥ : Set where

absurd : {A : Set} → ⊥ → A
absurd ()
\end{code}

\noindent
\textsc{Exercise 3.2}
\begin{itemize}
\item If $A$ then ($B$ implies $A$)
\begin{code}
p1 : {A B : Set} → A → (B → A)
p1 x = λ _ → x
\end{code}
\item If ($A$ and $\mathit{true}$) then ($A$ or $\mathit{false}$)
\begin{code}
p2 : {A : Set} → (A × ⊤) → (Either A ⊥)
p2 (x , tt) = left x
\end{code}
\item If $A$ implies ($B$ implies $C$), then ($A$ and $B$) implies $C$.
\begin{code}
p3 : {A B C : Set} → (A → (B → C)) → ((A × B) → C)
p3 f (x , y) = f x y
\end{code}
\item If $A$ and ($B$ or $C$), then either ($A$ and $B$) or ($A$ and $C$).
\begin{code}
p4 : {A B C : Set} → (A × (Either B C)) → (Either (A × B) (A × C))
p4 (x , left y) = left (x , y)
p4 (x , right z) = right (x , z)
\end{code}
\item If $A$ implies $C$ and $B$ implies $D$, then ($A$ and $B$) implies ($C$ and $D$).
\begin{code}
p5 : {A B C D : Set} → ((A → C) × (B → D)) → ((A × B) → (C × D))
p5 (f , g) (x , y) = f x , g y
\end{code}
\end{itemize}

\begin{code}
proof3 : {P Q R : Set} → (Either P Q → R) → (P → R) × (Q → R)
proof3 f = (λ x → f (left x)) , (λ x → f (right x))
\end{code}

\noindent
\textsc{Exercise 3.3}: Write a function of type \verb!{P : Set} → (Either P (P → ⊥) → ⊥) → ⊥!.

Assuming \verb!(Either P (P → ⊥) → ⊥)! then \texttt{proof3} above says that \verb!P → ⊥! and \verb!(P → ⊥) → ⊥!. Applying \verb!(P → ⊥) → ⊥! to \verb!P → ⊥! results in \verb!⊥! which proves the proposition.

\begin{code}
constructive-P-or-not-P : {P : Set} → (Either P (P → ⊥) → ⊥) → ⊥
constructive-P-or-not-P {P} f =
    (λ (x : P → ⊥) → f (right x)) (λ (x : P) → f (left x))
\end{code}

\noindent Some even stuff:

\begin{code}
data IsEven : Nat → Set where
    zeroIsEven : IsEven zero
    sucsucIsEven : {n : Nat} → IsEven n → IsEven (suc (suc n))

6-is-even : IsEven 6
6-is-even = sucsucIsEven (sucsucIsEven (sucsucIsEven zeroIsEven))

7-is-even : IsEven 7 → ⊥
7-is-even (sucsucIsEven (sucsucIsEven (sucsucIsEven ())))

data IsTrue : Bool → Set where
    TrueIsTrue : IsTrue true

_=Nat_ : Nat → Nat → Bool
zero =Nat zero = true
suc x =Nat suc y = x =Nat y
_ =Nat _ = false

length-is-3 : IsTrue (length (1 :: 2 :: 3 :: []) =Nat 3)
length-is-3 = TrueIsTrue
\end{code}

\begin{code}
double : Nat → Nat
double zero = zero
double (suc n) = suc (suc (double n))

double-is-even : (n : Nat) → IsEven (double n)
double-is-even zero = zeroIsEven
double-is-even (suc n) = sucsucIsEven (double-is-even n)
\end{code}

\begin{code}
n-equals-n : (n : Nat) → IsTrue (n =Nat n)
n-equals-n zero = TrueIsTrue
n-equals-n (suc n) = n-equals-n n

half-a-dozen : Σ Nat (λ n → IsTrue ((n + n) =Nat 12))
half-a-dozen = 6 , TrueIsTrue

zero-or-suc : (n : Nat) → Either
    (IsTrue (n =Nat zero))
    (Σ Nat (λ m → IsTrue (n =Nat (suc m))))
zero-or-suc zero = left TrueIsTrue
zero-or-suc (suc m) = right (m , n-equals-n m)
\end{code}

\noindent
\textsc{The Identity Type}

\begin{code}
data _≡_ {A : Set} : A → A → Set where
    refl : {x : A} → x ≡ x

infix 4 _≡_

n-equals-n-≡ : (n : Nat) → n ≡ n
n-equals-n-≡ n = refl

zero-not-one : 0 ≡ 1 → ⊥
zero-not-one ()
\end{code}

\noindent
I am curious about an equivalency that must take an argument:

\begin{code}
data _≡'_ {A : Set} : A → A → Set where
    refl : (x : A) → x ≡' x

infix 4 _≡'_

n-equals-n-≡' : (n : Nat) → n ≡' n
n-equals-n-≡' n = refl n

n-equals-n-≡'' : (n : Nat) → n ≡' n
n-equals-n-≡'' = refl
\end{code}

\noindent
Various laws of equivalency:

\begin{code}
sym : {A : Set} {x y : A} → x ≡ y → y ≡ x
sym refl = refl

trans : {A : Set} {x y z : A} → x ≡ y → y ≡ z → x ≡ z
trans refl refl = refl

cong : {A B : Set} {x y : A} → (f : A → B) → x ≡ y → f x ≡ f y
cong f refl = refl
\end{code}

\noindent
\textsc{Equational Reasoning}

\begin{code}
begin_ : {A : Set} → {x y : A} → x ≡ y → x ≡ y
begin p = p

_end : {A : Set} → (x : A) → x ≡ x
x end = refl

_≡⟨_⟩_ : {A : Set} → (x : A) → {y z : A}
    → x ≡ y → y ≡ z → x ≡ z
x ≡⟨ p ⟩ q = trans p q

_≡⟨⟩_ : {A : Set} → (x : A) → {y : A} → x ≡ y → x ≡ y
x ≡⟨⟩ q = x ≡⟨ refl ⟩ q

infix 1 begin_
infix 3 _end
infixr 2 _≡⟨_⟩_
infixr 2 _≡⟨⟩_
\end{code}

\noindent
Simple examples:

\begin{code}
[_] : {A : Set} → A → List A
[ x ] = x :: []

reverse : {A : Set} → List A → List A
reverse [] = []
reverse (x :: xs) = reverse xs ++ [ x ]

reverse-singleton : {A : Set} (x : A) → reverse [ x ] ≡ [ x ]
reverse-singleton x =
    begin
      reverse [ x ]
    ≡⟨⟩
      reverse (x :: [])
    ≡⟨⟩
      reverse [] ++ [ x ]
    ≡⟨⟩
      [] ++ [ x ]
    ≡⟨⟩
      [ x ]
    end
\end{code}

\noindent
Proof by induction and cases:

\begin{code}
add-n-zero : (n : Nat) → n + zero ≡ n
add-n-zero zero = refl
add-n-zero (suc n) =
    begin
      suc n + zero
    ≡⟨⟩
      suc (n + zero)
    ≡⟨ cong suc (add-n-zero n) ⟩
      suc n
    end
\end{code}

\noindent
\textsc{Exercise 4.1}: Prove that \verb!m + suc n = suc (m + n)! for all natural numbers $m$ and $n$. Next, use the previous lemma and this one to prove commutativity of addition, i.e. that $m + n = n + m$ for all natural numbers $m$ and $n$.

\begin{code}
add-suc : (m n : Nat) → m + suc n ≡ suc (m + n)
add-suc zero n = refl
add-suc (suc m) n = cong suc (add-suc m n)

add-commut : (m n : Nat) → m + n ≡ n + m
add-commut zero n = sym (add-n-zero n)
add-commut (suc m) n =
    begin
      suc m + n
    ≡⟨⟩
      suc (m + n)
    ≡⟨ cong suc (add-commut m n) ⟩
      suc (n + m)
    ≡⟨ sym (add-suc n m) ⟩
      n + suc m
    end

add-commut' : (m n : Nat) → m + n ≡ n + m
add-commut' zero n = sym (add-n-zero n)
add-commut' (suc m) n =
    trans (cong suc (add-commut' m n)) (sym (add-suc n m))
\end{code}

\begin{code}
add-assoc : (x y z : Nat) → x + (y + z) ≡ (x + y) + z
add-assoc zero y z = refl
add-assoc (suc x) y z = cong suc (add-assoc x y z)
\end{code}

\noindent
\textsc{Exercise 4.2}: Consider the following function:
\begin{code}
replicate : {A : Set} → Nat → A → List A
replicate zero x = []
replicate (suc n) x = x :: replicate n x
\end{code}
Prove that the length of \verb!replicate n x! is always equal to $n$, by induction on the number $n$.

\begin{code}
repl-length : {A : Set} → (n : Nat) → (x : A) → length (replicate n x) ≡ n
repl-length zero x = refl -- length [] ≡ zero
repl-length (suc n) x =
    begin
      length (replicate (suc n) x)
    ≡⟨⟩ -- Apply replicate
      length (x :: replicate n x)
    ≡⟨⟩ -- Apply length
      suc (length (replicate n x))
    ≡⟨ cong suc (repl-length n x) ⟩
      suc n
    end
\end{code}

\noindent
\textsc{Exercise 4.3}: The proofs of the lemmas:

\begin{code}
append-[] : {A : Set} → (xs : List A) → xs ++ [] ≡ xs
append-[] [] = begin ([] ++ []) ≡⟨⟩ [] end
append-[] (x :: xs) =
    begin
      (x :: xs) ++ []
    ≡⟨⟩ -- Apply ++
      x :: (xs ++ [])
    ≡⟨ cong (x ::_) (append-[] xs) ⟩
      x :: xs
    end

append-assoc : {A : Set} → (xs ys zs : List A)
    → (xs ++ ys) ++ zs ≡ xs ++ (ys ++ zs)
append-assoc [] ys zs =
    begin
      ([] ++ ys) ++ zs
    ≡⟨⟩ -- Apply ++
      ys ++ zs
    ≡⟨⟩ -- Unapply ++
      [] ++ (ys ++ zs)
    end
append-assoc (x :: xs) ys zs =
    begin
      ((x :: xs) ++ ys) ++ zs
    ≡⟨⟩ -- Apply ++
      (x :: (xs ++ ys)) ++ zs
    ≡⟨⟩ -- Apply ++
      x :: ((xs ++ ys) ++ zs)
    ≡⟨ cong (x ::_) (append-assoc xs ys zs) ⟩
      x :: (xs ++ (ys ++ zs))
    ≡⟨⟩ -- Unapply ++
      (x :: xs) ++ (ys ++ zs)
    end
\end{code}

\noindent
Now we can prove distributivity of reverse:

\begin{code}
reverse-distributivity : {A : Set} → (xs ys : List A)
    → reverse (xs ++ ys) ≡ reverse ys ++ reverse xs
reverse-distributivity [] ys =
    begin
      reverse ([] ++ ys)
    ≡⟨⟩ -- Apply ++
      reverse ys
    ≡⟨ sym (append-[] (reverse ys)) ⟩ -- Use the append-[] lemma
      reverse ys ++ []
    ≡⟨⟩ -- Unapply reverse to []
      reverse ys ++ reverse []
    end
\end{code}
\begin{code}
reverse-distributivity (x :: xs) ys =
    begin
      reverse ((x :: xs) ++ ys)
    ≡⟨⟩ -- Apply ++
      reverse (x :: (xs ++ ys))
    ≡⟨⟩ -- Apply reverse
      reverse (xs ++ ys) ++ [ x ]
    ≡⟨ cong (_++ [ x ]) (reverse-distributivity xs ys) ⟩
      (reverse ys ++ reverse xs) ++ [ x ]
    ≡⟨ append-assoc (reverse ys) (reverse xs) [ x ] ⟩
      reverse ys ++ (reverse xs ++ [ x ])
    ≡⟨⟩ -- Unapply reverse
      reverse ys ++ reverse (x :: xs)
    end
\end{code}

\noindent
And that reversing twice is idempotent:

\begin{code}
reverse-reverse : {A : Set} → (xs : List A) → reverse (reverse xs) ≡ xs
reverse-reverse [] = refl
reverse-reverse (x :: xs) =
    begin
      reverse (reverse (x :: xs))
    ≡⟨⟩ -- Apply inner reverse
      reverse (reverse xs ++ [ x ])
    ≡⟨ reverse-distributivity (reverse xs) [ x ] ⟩
      reverse [ x ] ++ reverse (reverse xs)
    ≡⟨ cong ([ x ] ++_) (reverse-reverse xs) ⟩
      [ x ] ++ xs
    ≡⟨⟩ -- Apply ++
      x :: xs
    end
\end{code}

\noindent
Functor laws for lists:

\begin{code}
map-id : { A : Set } → (xs : List A) → map id xs ≡ xs
map-id [] =
  begin
    map id []
  ≡⟨⟩ -- Apply map
    []
  end
map-id (x :: xs) =
  begin
    map id (x :: xs)
  ≡⟨⟩ -- Apply map and id
    x :: (map id xs)
  ≡⟨ cong (x ::_) (map-id xs) ⟩
    x :: xs
  end
\end{code}

\begin{code}
_◦_ : {A B C : Set} → (B → C) → (A → B) → (A → C)
f ◦ g = λ x → f (g x)

infixr 9 _◦_

map-compose : {A B C : Set} → (f : B → C) → (g : A → B)
    → (xs : List A) → map (f ◦ g) xs ≡ map f (map g xs)
map-compose f g [] = refl
map-compose f g (x :: xs) =
  begin
    map (f ◦ g) (x :: xs)
  ≡⟨⟩ -- Apply map
    (f ◦ g) x :: map (f ◦ g) xs
  ≡⟨ cong (f (g x) ::_) (map-compose f g xs) ⟩ -- Also apply ◦
    f (g x) :: map f (map g xs)
  ≡⟨⟩ -- Unapply both maps
    map f (map g (x :: xs))
  end
\end{code}

\noindent
\textsc{Exercise 4.4}: Prove that \verb!length (map f xs)! is equal to \verb!length xs! for all \verb!xs!.

\begin{code}
map-length : {A B : Set} → (f : A → B) → (xs : List A)
    → length (map f xs) ≡ length xs
map-length f [] = refl
map-length f (x :: xs) = cong suc (map-length f xs)
\end{code}

\noindent
\textsc{Exercise 4.4}: Define the functions \texttt{take} and \texttt{drop} that respectively return or remove the first $n$ elements of the list (or all elements if the list is shorter). Prove that for any number $n$ and any list \verb!xs!, we have \verb!take n xs ++ drop n xs! = \verb!xs!.

\begin{code}
take : {A : Set} → Nat → List A → List A
take zero _ = []
take _ [] = []
take (suc n) (x :: xs) = x :: (take n xs)

drop : {A : Set} → Nat → List A → List A
drop zero xs = xs
drop _ [] = []
drop (suc n) (x :: xs) = drop n xs
\end{code}

\begin{code}
take-drop : {A : Set} → (n : Nat) → (xs : List A)
    → take n xs ++ drop n xs ≡ xs
take-drop zero xs =
  begin
    take zero xs ++ drop zero xs
  ≡⟨⟩ -- Apply take and drop
    [] ++ xs
  ≡⟨⟩ -- Apply ++
    xs
  end
take-drop (suc n) [] =
  begin
    take (suc n) [] ++ drop (suc n) []
  ≡⟨⟩ -- Apply take and drop
    [] ++ []
  ≡⟨⟩ -- Apply ++
    []
  end
take-drop (suc n) (x :: xs) =
  begin
    take (suc n) (x :: xs) ++ drop (suc n) (x :: xs)
  ≡⟨⟩ -- Apply take and drop
    (x :: (take n xs)) ++ drop n xs
  ≡⟨⟩ -- Apply ++
    x :: (take n xs ++ drop n xs)
  ≡⟨ cong (x ::_) (take-drop n xs) ⟩
    x :: xs
  end
\end{code}

\noindent
Finally back to Hutton! Note, I am following more the exposition from Hutton than from Cockx.

\begin{code}
reverse' : {A : Set} → List A → List A → List A
reverse' [] ys = ys
reverse' (x :: xs) ys = reverse' xs (x :: ys)
\end{code}

\begin{code}
reverse'-reverse : {A : Set} → (xs ys : List A)
    → reverse' xs ys ≡ reverse xs ++ ys
reverse'-reverse [] ys = refl
reverse'-reverse (x :: xs) ys =
  begin
    reverse' (x :: xs) ys
  ≡⟨⟩ -- Apply reverse'
    reverse' xs (x :: ys)
  ≡⟨ reverse'-reverse xs (x :: ys) ⟩
    reverse xs ++ (x :: ys)
  ≡⟨⟩ -- Unapply ++
    reverse xs ++ ([ x ] ++ ys)
  ≡⟨ sym (append-assoc (reverse xs) [ x ] ys) ⟩
    (reverse xs ++ [ x ]) ++ ys
  ≡⟨⟩ -- Unapply reverse
    reverse (x :: xs) ++ ys
  end
\end{code}

\begin{code}
reverse'-reverse-equiv : {A : Set} → (xs : List A)
    → reverse' xs [] ≡ reverse xs
reverse'-reverse-equiv xs =
  begin
    reverse' xs []
  ≡⟨ reverse'-reverse xs [] ⟩
    reverse xs ++ []
  ≡⟨ append-[] (reverse xs) ⟩
    reverse xs
  end
\end{code}

\begin{code}
data Tree (A : Set) : Set where
    Leaf : A → Tree A
    Node : Tree A → Tree A → Tree A

flatten : {A : Set} → Tree A → List A
flatten (Leaf x) = [ x ]
flatten (Node tl tr) = flatten tl ++ flatten tr

flatten' : {A : Set } → Tree A → List A → List A
flatten' (Leaf x) xs = x :: xs
flatten' (Node tₗ tᵣ) xs = flatten' tₗ (flatten' tᵣ xs)

flatten'-flatten : {A : Set} → (t : Tree A) → (xs : List A)
    → flatten' t xs ≡ flatten t ++ xs
flatten'-flatten (Leaf x) xs = refl
flatten'-flatten (Node tₗ tᵣ) xs =
  begin
    flatten' (Node tₗ tᵣ) xs
  ≡⟨⟩ -- Apply flatten'
    flatten' tₗ (flatten' tᵣ xs)
  ≡⟨ cong (flatten' tₗ) (flatten'-flatten tᵣ xs) ⟩
    flatten' tₗ (flatten tᵣ ++ xs)
  ≡⟨ flatten'-flatten tₗ (flatten tᵣ ++ xs) ⟩
    flatten tₗ ++ (flatten tᵣ ++ xs)
  ≡⟨ sym (append-assoc (flatten tₗ) (flatten tᵣ) xs) ⟩
    (flatten tₗ ++ flatten tᵣ) ++ xs
  ≡⟨⟩ -- Unapply flatten
    flatten (Node tₗ tᵣ) ++ xs
  end
\end{code}

\end{document}
